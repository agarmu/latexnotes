\usepackage[utf8]{inputenc}
\usepackage[T1]{fontenc}
\usepackage{textcomp}
\usepackage{url}
\usepackage{hyperref}
\hypersetup{
    colorlinks,
     linkcolor={black},
     citecolor={black},
     urlcolor={blue!80!black},
     linktoc=all
}
\usepackage{graphicx}
\usepackage{float}
\usepackage{booktabs, longtable}
\usepackage[shortlabels]{enumitem}
% \usepackage{parskip}
\usepackage{emptypage}
\usepackage{subcaption}
\usepackage{multicol}
\usepackage[cmyk,usenames,dvipsnames,table]{xcolor}

% Format/layout
\usepackage{layout}
\usepackage{geometry}
\geometry{
    letterpaper,
    inner=1in,
    top=1in,
    outer=1in,
    bottom=1in,
    marginparwidth=0.5in
}
% No indentation
\setlength\parindent{0pt}


\definecolor{correct}{HTML}{009900}
\newcommand\correct[2]{\ensuremath{\:}{\color{red}{#1}}\ensuremath{\to }{\color{correct}{#2}}\ensuremath{\:}}
\newcommand\green[1]{{\color{correct}{#1}}}

\usepackage{fancyhdr, titling, lastpage, etoolbox}
\pagestyle{fancy}
\fancyhf{}
\fancyfoot[LE, RO]{\thepage}% Custom footer
\renewcommand{\headrulewidth}{0.4pt}% Line at the header visible
\renewcommand{\footrulewidth}{0.4pt}% Line at the footer visible

% Set chapter page style
  \fancypagestyle{chapterStyle}{
      \fancyhf{}
      \fancyfoot[LE, RO]{\thepage}% Custom footer
      \renewcommand{\headrulewidth}{0pt}% Line at the header not visible
      \renewcommand{\footrulewidth}{0.4pt}% Line at the footer visible
}

\patchcmd{\chapter}{\thispagestyle{plain}}{\thispagestyle{chapterStyle}}{}{}

\usepackage{libertine}
\usepackage{libertinust1math}

\usepackage{titlesec}
\titleformat{\chapter}{\centering\bfseries\Huge}{\thechapter}{0.5em}{}

%%%%%%%%%%%%%%%%
%%    Math    %%
%%%%%%%%%%%%%%%%

\usepackage{amsmath, amsfonts, mathtools, amsthm, amssymb}
\usepackage{cancel}
\usepackage{bm}

% Sets
\newcommand\N{\ensuremath{\mathbb{N}}} % Natural Numbers
\newcommand\Z{\ensuremath{\mathbb{Z}}} % Integers
\newcommand\D{\ensuremath{\mathbb{D}}} % Decimal Numbers
\newcommand\Q{\ensuremath{\mathbb{Q}}} % Rational Numbers
\newcommand\R{\ensuremath{\mathbb{R}}} % Real Numbers
\newcommand\C{\ensuremath{\mathbb{C}}} % Complex Numbers
\renewcommand\O{\ensuremath{\emptyset}} % Empty Set

\usepackage{systeme} % Systems of eqns.

\let\implies\Rightarrow
\let\impliedby\Leftarrow
\let\iff\Leftrightarrow


% horizontal rule
\newcommand\hr{
    \noindent\rule[0.5ex]{\linewidth}{0.5pt}
}


% hide parts
\newcommand\hide[1]{}



% si unitx
\usepackage{siunitx}
\sisetup{locale = US}
\newcommand\mat[1]{\mathbf{#1}}

% colored box environments

\usepackage{tikz}
\usepackage{tikz-cd}
\usetikzlibrary{intersections, angles, quotes, calc, positioning}
\usetikzlibrary{arrows.meta}
\usepackage{pgfplots}
\pgfplotsset{compat=1.13}
\tikzset{
    force/.style={thick, {Circle[length=2pt]}-stealth, shorten <=-1pt}
}



\usepackage{thmtools, tcolorbox}
\usepackage[framemethod=TikZ]{mdframed}
\mdfsetup{skipabove=1em,skipbelow=0em}


\tcbuselibrary{theorems, breakable}


\newtcbtheorem[no counter]{definition}{Definition}%
{separator sign={\hspace{-0.25em}:},breakable,colback=ForestGreen!5,colframe=ForestGreen,fonttitle=\bfseries\sffamily,before skip=7.5pt,after skip=7.5pt}{def}

\newtcbtheorem[number within=chapter]{prop}{Proposition}%
{breakable,colback=Emerald!15,colframe=Emerald,fonttitle=\bfseries\sffamily,before skip=7.5pt,after skip=7.5pt}{prop}

\newtcbtheorem[number within=chapter]{theorem}{Theorem}%
{separator sign={}, breakable,colback=Emerald!15,colframe=Emerald,fonttitle=\bfseries\sffamily,before skip=7.5pt,after skip=7.5pt}{thm}

\newtcbtheorem[number within=chapter]{lemma}{Lemma}%
{separator sign={}, breakable,colback=Emerald!15,colframe=Emerald,fonttitle=\bfseries\sffamily,before skip=7.5pt,after skip=7.5pt}{lem}

\declaretheoremstyle[
    headfont=\bfseries\sffamily\color{Emerald!70!black}, bodyfont=\normalfont,
    mdframed={
        linewidth=2pt,
        rightline=false, topline=false, bottomline=false,
        linecolor=Emerald, backgroundcolor=Emerald!15,
        skipabove=7.5pt,skipbelow=7.5pt,
    }
]{corollarybox}
\declaretheorem[style=corollarybox, numbered=no, name=Corollary]{corollary}

\newtcbtheorem[no counter]{law}{Law}%
{separator sign={\hspace{-0.25em}:}, breakable,colback=Plum!15,colframe=Plum,fonttitle=\bfseries\sffamily,before skip=7.5pt,after skip=7.5pt}{cor}

\newtcbtheorem[no counter]{formula}{Formula}%
{separator sign={\hspace{-0.25em}:}, breakable,colback=Violet!15,colframe=Violet,fonttitle=\bfseries\sffamily,before skip=7.5pt,after skip=7.5pt}{form}

\declaretheoremstyle[
    headfont=\bfseries\sffamily\color{WildStrawberry!70!black}, bodyfont=\normalfont,
    mdframed={
        linewidth=2pt,
        rightline=false, topline=false, bottomline=false,
        linecolor=WildStrawberry, backgroundcolor=WildStrawberry!15,
        skipabove=7.5pt,skipbelow=7.5pt,
    }
]{examplebox}
\declaretheorem[style=examplebox, numbered=no, name=Example]{eg}

\declaretheoremstyle[
    headfont=\bfseries\sffamily\color{Melon!70!black}, bodyfont=\normalfont,
    mdframed={
        linewidth=2pt,
        rightline=false, topline=false, bottomline=false,
        linecolor=Melon, backgroundcolor=Melon!15,
        skipabove=7.5pt,skipbelow=7.5pt,
    }
]{explanationbox}
\declaretheorem[style=explanationbox, numbered=no, name=Explanation]{explanation}

\declaretheoremstyle[
    headfont=\bfseries\sffamily\color{NavyBlue!70!black}, bodyfont=\normalfont,
    numbered=no,
    mdframed={
        linewidth=2pt,
        rightline=false, topline=false, bottomline=false,
        linecolor=NavyBlue, backgroundcolor=NavyBlue!15,
        skipabove=-7.5pt, skipbelow=7.5pt
    },
]{thmproofbox}
\declaretheorem[style=thmproofbox, name=Proof]{replacementproof}
\renewenvironment{proof}[1][\proofname]{\begin{replacementproof}}{\end{replacementproof}}


\declaretheoremstyle[
    headfont=\bfseries\sffamily\color{Orange}, bodyfont=\normalfont,
    mdframed={
        linewidth=2pt,
        rightline=false, topline=false, bottomline=false,
        linecolor=Orange, backgroundcolor=Orange!25
    }
]{thmorangebox}
\declaretheoremstyle[
    headfont=\bfseries\sffamily\color{Dandelion}, bodyfont=\normalfont,
    mdframed={
        linewidth=2pt,
        rightline=false, topline=false, bottomline=false,
        linecolor=Dandelion, backgroundcolor=Dandelion!25
    }
]{thmdandelionbox}
\declaretheorem[style=thmdandelionbox, numbered=no, name=Remark]{remark}
\declaretheorem[style=thmorangebox, numbered=no, name=Note]{note}

% figure support
\usepackage{import}
\usepackage{xifthen}
\pdfminorversion=7
\usepackage{pdfpages}
\usepackage{transparent}
\newcommand{\incfig}[1]{%
    \def\svgwidth{\columnwidth}
    \import{./figures/}{#1.pdf_tex}
}

% %http://tex.stackexchange.com/questions/76273/multiple-pdfs-with-page-group-included-in-a-single-page-warning
\pdfsuppresswarningpagegroup=1
